%! Author = IlBaia
%! Date = 22/06/2022

\documentclass{article}
\usepackage[utf8]{inputenc}
\usepackage[T1]{fontenc}
\usepackage[margin=1in]{geometry}
\usepackage[italian]{babel}
\usepackage{graphicx}
\usepackage{float}
\usepackage{lipsum}
\usepackage{longtable}
\usepackage[section]{placeins}
\usepackage{hyperref} %import hyperref
\graphicspath{{graphics/}}
\hypersetup{
    colorlinks=true,
    linkcolor=blue,
    filecolor=magenta,
    urlcolor=magenta,
    pdftitle={Multilayered Perceptron},
    pdfpagemode=FullScreen,
    }

\title{Multilayered Perceptron}
\author{Lorenzo Baiardi}
\date{Giugno 27, 2022}

% Document
\begin{document}

\maketitle

    \section{Introduzione}\label{sec:introduzione}
        In questa relazione verrà trattato il multilayered perceptron, una rete in grado di apprendere dai dati in input,
        detti features, in valori di uscita (prediction).
    \section{MLP}\label{sec:tipologia}
        In questo problema stiamo considerando un multilayered Perceptron in grado di predire più di una classe in output.
    \subsection{Softmax}\label{subsec:softmax}
    \subsection{Metodo del gradiente stocastico}\label{subsec:metodo-del-gradiente-stocastico}
    \section{Dataset}\label{sec:dataset}
    \section{Valori Attesi}\label{sec:valori-attesi}
    \section{Risultati}\label{sec:risultati}
    \section{Conclusioni}\label{sec:conclusioni}
    \section{Caratteristiche Terminale}\label{sec:caratteristiche-terminale}
        Il terminale su cui sono state svolte le prove:
        \begin{itemize}
            \item Sistema Operativo: Windows 11 PRO
            \item Processore: Intel I5-8600K
            \item Scheda Video: Nvdia GEFORCE 1050-Ti
            \item Ram: 16GB DDR4 3600MHz
        \end{itemize}
    \section{Riferimenti}\label{sec:riferimenti}

\end{document}